\documentclass{article}
\usepackage{amsmath,amssymb,amsfonts}
\usepackage{graphicx}
\usepackage{hyperref}

\title{Math Notes: fundamentals}
\author{trelly}
\date{\today}

\begin{document}

\section*{fundamentals} 
\section{Key Concepts}
\section{Numbers}
Numbers are the basic objects we use to count, measure, quantify,
and calculate things. Mathematicians like to classify the different
kinds of number-like objects into categories called sets:
\begin{\begin{itemize}
  \item natural numbers:  N “{0, 1, 2, 3, 4, 5, 6, 7, . . .}  
  \item integers: Z “ t. . . , ´3, ´2, ´1, 0, 1, 2, 3, . . . u
  \item rational numbers: Q “ t 5
3 , 22
7 , 1.5, 0.125, ´7, . . . u 
\item  The real numbers: R “ t´1, 0, 1, ?2, e, π, 4.94 . . . , . . . u
  \ complex numbers  C “ t´1, 0, 1, i, 1 ` i, 2 ` 3i, . . 
\end{itemize}}

\section{Operator precedence}
 1. Parentheses
 2. Exponents
3. Multiplication and Division
4. Addition and Subtraction

\section{Practice Problems}

also known as PEMDAS  the order of operations baby
\section{Quadreatic expression}
\section{Cartesian plane}
\section{Definitions}
% Important definitions go here

\section{Formulas and Equations}
% Key formulas and equations

\section{Examples}
% Worked examples or sample problems

\section{Practice Problems}
% Problems to solve or questions to answer

x^\sqrt{2} - 4 = 45
\end{document}
