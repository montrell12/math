\documentclass{article}
\usepackage{amsmath, amssymb, amsfonts}
\usepackage{graphicx}
\usepackage{hyperref}
\usepackage{bookmark}
\usepackage{pgfplots}
\usepackage{xcolor}  % Added package for color
\pgfplotsset{compat=1.18}

% Set the background color to tan
\usepackage{pagecolor}  % Use pagecolor to set background
\definecolor{tan}{rgb}{0.82, 0.71, 0.55}  % Define tan color (adjust as needed)
\pagecolor{tan}  % Apply the color to the entire page

\title{Trig Notes}
\author{Trell}
\date{04/28/2025}  % Empty date

\begin{document}

\maketitle

\section{Notes}

\begin{itemize}
    \item \textbf{What is Pi?} \\
    The symbol \( \pi \) (pi) represents the ratio of a circle's circumference to its diameter. In other words, no matter the size of the circle, the circumference (the distance around the circle) is always approximately 3.14159 times the length of the diameter (the distance across the circle through its center).

    \item \textbf{Diameter} \\
    The diameter is the distance across the circle through its center. It can go in any direction, but it must pass through the center of the circle. \\
    \textit{Word origin:} The term comes from the Greek word "diametros" (διάμετρος), which is composed of two parts:
    \begin{itemize}
        \item "dia-" (διά) meaning "through" or "across."
        \item "metron" (μέτρον) meaning "measure."
    \end{itemize}
    Thus, diameter literally means "measure across" or "measure through." It refers to the measurement of the straight line passing through the center of a circle, connecting two points on the circumference.

    \item \textbf{Radian} \\
    Imagine a circle with a radius \( r \). If you take an arc (a part of the circle's circumference) whose length is exactly equal to the radius of the circle, the angle between the two radii that intersect at the endpoints of the arc is exactly 1 radian. \\
    In simpler terms, 1 radian is the angle formed when you "unwrap" the length of the radius along the circumference of the circle.

    \item \textbf{Radius} \\
    The radius is the distance from the center of a circle or sphere to any point on its circumference or surface. It is essentially half the length of the diameter. In geometry, the radius is a key measurement used to define the size of a circle or sphere.

    \item \textbf{Coterminal Angles} \\
    Two angles are coterminal if they share the same terminal side. For example, \( \frac{5}{4} \pi \) and \( \frac{3}{4} \pi \) are coterminal because they differ by \( 2\pi n \), where \( n \) is any integer.

    \item \textbf{Hypotenuse} \\
    The hypotenuse is the longest side of a right-angled triangle, opposite the right angle (90°). It is a key concept in trigonometry and the Pythagorean theorem. The hypotenuse always represents the radius of a circle when the triangle is inscribed in the circle.

    
  \item \textbf{Pythagorean Theorem} - The Pythagorean Theorem is a fundamental principle in geometry that relates the sides of a right-angled triangle. It states:
    In a right triangle, the square of the hypotenuse (the side opposite the right angle) is equal to the sum of the squares of the other two sides (legs).
    a,b = lengths of the legs (shorter sides),
    c = length of the hypotenuse (longest side).
    Pythagorean theorem: $c^2 = a^2 + b^2 $.  

    \item \textbf{Degrees and Radians} \\
    For an angle \( \theta \), when \( \theta = 60^\circ \), we can express this angle in radians as:
    \[
    \theta = \frac{\pi}{3} \text{ radians.}
    \]
\end{itemize}

\end{document}
